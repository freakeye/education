\documentclass[a4paper, 12pt]{article}

%\usepackage[T1]{fontenc}
\usepackage[T2A]{fontenc}
\usepackage[utf8]{inputenc}
\usepackage[russian]{babel}

\usepackage{mlmodern}
\usepackage{amssymb, mathtools}
\usepackage{pb-diagram}
%\usepackage{multicol}

\textheight=25cm
\textwidth=18cm
\voffset=-3cm
\hoffset=-2cm
\title{СВОБОДНЫЙ УНИВЕРСИТЕТ}
\date{}
\author{Алексей Мячин}

\input{glyphtounicode}
\pdfgentounicode=1


\begin{document}
\maketitle
\section*{ Основные определения курса в терминах универсальной конструкции }

{\quad}

Векторные пространства (ВП) ниже предполагаются конечномерными, категория $\mathcal{V} = Vct = Vect^{fin}_\Bbbk$.

\subsection*{Базисы векторных пространств}

Возьмём две категории $\mathcal{C} = Vct, \mathcal{D} = Set$ и функтор $F$ -- забывающий структуру векторного пространства $F: \mathcal{C} \to \mathcal{D}$ -- от каждого ВП он оставляет множество векторов.

Построим универсальный морфизм из произвольного фиксированного множества $X \in \mathcal{D}$ для функтора $F$.

Для множества $X$ образуем множество формальных (т.е. вектор -- это просто запись, без вычислений и результата) конечных линейных комбинаций c коэффициентами в $\Bbbk$: $V_X := Span_\Bbbk(X)$ -- это ВП с базисом ровно из векторов -- элементов множества $X$.
Т.о. в категории $\mathcal{D}$ определена стрелка $u: X \to F(V_X)$, которая множеству неких элементов ставит в соответствие то же множество, но уже рассматриваемое как множество векторов в базисе $V_X$.

\medskip
\underline{Утверждение}. Пара $(u, X)$ имеет универсальное свойство для функтора $F$.

Возьмём произвольное ВП $A'$, тогда определено множество $F(A')$.

Пусть есть стрелка $f: X \to F(A')$. Покажем как можно пропустить $f$ через пару $(u, X)$.

Для множества $X$ построим $V_X$. Теперь в категории $Vct$ можем построить линейное отображение $h: V_X \to A'$. Оно опеределено однозначно, т.к. его значения в $A'$ уже заданы (отображением $f$) на базисе $X$. Далее с базиса оно только продолжается по линейности на $V_X$, а такое продолжение единственно (любой вектор $V_X$ записывается в базисе единственным способом).

Перенеся $h$ функтором $F$, получим
$$ F(h)\circ u = f $$

\indent
\begin{minipage}[t]{40mm}\parindent=2em
В категории $\mathcal{C}$

\[ \begin{diagram}
	\node{V_X}
		\arrow{s,l}{h}
	\\
	\node{A'}	
\end{diagram} \]

\end{minipage}
\hfill
\begin{minipage}[t]{60mm}
В категории $\mathcal{D}$

\[ \begin{diagram}
	\node{X}
		\arrow[2]{e,t}{u}
		\arrow{se,b}{f}
	\node[2]{F(V_X)}
		\arrow{sw,r}{F(h)}
	\\
	\node[2]{F(A')}	
\end{diagram} \]

\end{minipage}


\subsection*{Декартово произведение ВП}

В категории $\mathcal{V}$, зафиксируем пару ВП $V_1, V_2$.
С их участием построим контравариантный функтор $F: \mathcal{V}^{op} \to Set$ с действием
\begin{itemize}
\item на объектах: $U \mapsto Hom(U, V_1) \times Hom(U, V_2)$, т.е. ВП $U$ переходит в множество отображений с элементами -- парами стрелок $U \to V_1$ и $U \to V_2$;
\item на стрелках: морфизм $l: U \to W$ переходит в морфизм в Set $Hom(W, V_1)\times Hom(W, V_2) \to Hom(U, V_1)\times Hom(U, V_2)$ (направление противоположно направлению $l$).
\end{itemize}

Образуем ВП $V_1 \times V_2$ -- новый объект в $\mathcal{V}$ и рассмотрим соответствующий ему объект -- пару отображений в Set: $(\pi_1 \times \pi_2) \in Hom(V_1 \times V_2, V_1) \times Hom(V_1 \times V_2, V_2), \;\; \pi_i: V_1 \times V_2 \to V_i$.

\medskip
\underline{Утверждение}. Пара $(V_1 \times V_2, \pi_1 \times \pi_2)$ имеет универсальное свойство для функтора $F$.

Возьмём другой морфизм в Set $f: F(W) \to V_1 \times V_2$ для какого-то ВП $W$. Множество $F(W)$ -- это пары отображений $(f_1, f_2), \; f_i : W \to V_i$.

Покажем что существует единственный морфизм $Fh: F(W) \to F(V_1 \times V_2)$ такой что
$$
f = (\pi_1 \times \pi_2) \circ Fh \Leftrightarrow f_i = \pi_i \circ Fh
$$

\[ \begin{diagram}
	\node[2]{F(W)}
		\arrow{s,r}{Fh}
		\arrow{sw,t}{f_1}
		\arrow{se,t}{f_2}
	\\
	\node{V_1}
	\node{V_1 \times V_2}
		\arrow{w,b}{\pi_1}
		\arrow{e,b}{\pi_2}
	\node{V_2}
\end{diagram} \]

Такой морфизм построим для данных $(f_1, f_2) = f$ как декартову пару из этих отображений $Fh = f_1\times f_2$ -- линейно по каждому компоненту, поэтому $f_1\times f_2$ является морфизмом $Hom(W, V_1) \times Hom(W, V_2) \to Hom(V_1 \times V_2, V_1) \times Hom(V_1 \times V_2, V_2)$ и, значит, определён в Set. Если теперь взять $\pi_i \circ Fh$, то есть просто оставить первый компонент пары, то получим то же самое отображение $f_i$ из $Hom(V_1 \times V_2, V_1)$.\\
Построенная стрелка $Fh$ также единственна, т.к. однозначно задаётся своими компонентами $f_i$.

\subsection*{Прямая сумма ВП}

С фиксированными ВП $V_1, V_2$ можно построить двойственную конструкцию -- копроизведение ВП в $\mathcal{V}$.
Для этого обратим стрелки и возьмём ковариантный функтор $F: \mathcal{V} \to Set$ с действием $U \mapsto Hom(V_1, U) \times Hom(V_2, U)$.

\medskip
\underline{Утверждение}. Пара $(V_1 \oplus V_2, f_1\oplus f_2)$ -- универсальный элемент функтора $F$. Т.е. для любой пары морфизмов ВП как элемента $Hom(V_1, U) \times Hom(V_2, U)$
\[ \begin{diagram}
	\node{V_1}
		\arrow{e,t}{f_1}
	\node{W}
	\node{V_2}
		\arrow{w,t}{f_2}
\end{diagram} \]
есть единственный морфизм векторных пространств
$$
h: V_1 \oplus V_2 \to W :=
\begin{cases}
	f_1, & \text{если $v \in V_1$;} \\
	f_2, & \text{если $v \in V_2$,}
\end{cases}
$$
такой что $Fh$ замыкает диаграмму (точнее обе её части):

\[ \begin{diagram}
	\node[2]{F(W)}
	\\
	\node{V_1}
		\arrow{ne,t}{f_1}
		\arrow{e,b}{\iota_1}
	\node{V_1 \oplus V_2}
		\arrow{n,r}{Fh}
	\node{V_2}
		\arrow{nw,t}{f_2}
		\arrow{w,b}{\iota_2}
\end{diagram} \]

$$
Fh \circ \iota_1 = f_1 \;\;\;
Fh \circ \iota_2 = f_2
$$

Если существуют какие-то стрелки $f_1, f_2$, то они однозначно задают морфизм $Fh$.\\
С точностью до изоморфизма, коммутирующего со стрелками $\iota_1, \iota_2$ диаграмма единственна, т.е. в категории $\mathcal{V}$ произведение $V_1 \times V_2 = V_1 \oplus V_2$.


\subsection*{Тензорное произведение}

В категории $Vct$ возьмём два объекта $U, V$ и рассмотрим функтор $F: Vct \to Set$, такой что (на объектах) любому ВП $W$ ставится в соответствие множество $Bilin(U,V; W)$ всех билинейных отображений $U \times V \to W$, т.е. линейных по обоим аргументам в отдельности $f(su, v) = f(u, sv) = sf(u,v)$.

Пусть $s \in \Bbbk$, $f: U \times V \to W$ -- линейное отображение ВП. Тогда для (прямого) декартового произведения ВП $U \times V$ есть 3 разных вектора:
$$f(su, v) \;\;\;\; f(u, sv) \;\;\;\; f(su, sv)$$
Аналогично, если проверять аддитивность.

Построим тензорное произведение ВП $U \otimes V$ с помощью билинейного отображения -- проекции на фактор-пространство $U \otimes V = U \times V / Ident $: векторы отображаются в класс эквивалентности $u \otimes v$, который содержит оба вектора выше и ещё два вектора для аддитивности.

Формально, выберем базис $\{(u_i, v_j) =: e_{u_i, v_j} | 1 \leq i \leq \dim U, \, 1 \leq j \leq \dim V \}$, порождающий $U \times V$. Рассмотрим линейную оболочку
\begin{equation*}
\begin{split}
Ident: = \langle e_{su_i, v_j} - se_{u_i, v_j},  	\,
				e_{u_i, sv_j} - se_{u_i, v_j},		\,
				e_{u_i + u', v_j} - e_{u_i, v_j} - e_{u', v_j},	\,
				e_{u_i, v_j + v'} - e_{u_i, v_j} - e_{v', v_j},	\; \\
			\forall{s} \in \Bbbk, \, u' \in U, \, v' \in V \rangle
\end{split}
\end{equation*}
и профакторизуем по ней
$$
U \otimes V =  U \times V / Ident
$$
Т.о. на отдельных векторах (тензорное произведение векторов) оно вынуждено быть билинейным:
$$\tau: (u, v) \mapsto u \otimes v$$

\medskip
\underline{Утверждение}. Пара $(U \otimes V, \tau)$ -- универсальный элемент для функтора $F = Bilin(U,V; \cdot)$

Возьмём произвольное билинейное отображение $f: U \times V \to W$, пропустим её через объект $U \otimes V$ с помощью единственной для данного $f$ стрелки $h_f$. При 
том $h_f$ линейно на $U \otimes V$:

\[ \begin{diagram}
	\node{U \times V}
		\arrow[2]{e,t}{\tau}
		\arrow{se,b}{f}
	\node[2]{U \otimes V}
		\arrow{sw,r}{h_f}
	\\
	\node[2]{W}	
\end{diagram} \]

В композиции $h_f \circ \tau$ линейное отображение применяется к билинейному, поэтому итоговое отображение будет билинейным: $h_f(\tau(u, sv)) = h_f(s\tau(u, v)) = sh_f(\tau(u, v))$.

Любой вектор пространства $U \otimes V$ есть линейная комбинация разложимых тензоров $u \otimes v$ для каких-то $u \in U, v \in V$. Поэтому $h_f$ определяется своими значениями на $u \otimes v$, каждый из которых содержится в образе $\tau$.

Теперь, по данному билинейному $f$ определяем значения линейного $h_f$ на разложимых тензорах: $h_f(u \otimes v) = f(u, v)$. Получаем, что $f$ и $ h_f \circ \tau$ совпадают как билинейные отображения.


\subsection*{Тензорная алгебра}
Всякая алгебра над полем $\Bbbk$ является векторным пространством над тем же полем.\\
Строим тензорную алгебру:
$$ T^0(V) = \Bbbk, \; T^1(V) = V, \; ... \; T^n(V) = V^{\otimes n} $$

Отображение $i_V: V=T^1(V) \to T(V)$ -- каноническое вложение.

\underline{Утверждение}. Для любой алгебры $A$ и любого линейного отображения $f: V \to A$ существует единственный гомоморфизм алгебр $h: T(V) \to A$ такой, что $h \circ i_V = f$.

\subsection*{Пара сопряжённых функторов}

\subsection*{Свободные структуры -- группы, алгебры и модули над кольцом}

\end{document}
