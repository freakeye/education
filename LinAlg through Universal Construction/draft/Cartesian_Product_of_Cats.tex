\subsection*{Декартово произведение ВП}

Работаем в категории $\mathcal{V}_\times$, объекты которой -- пары объектов $(V_1, V_2)$ взятые из двух категорий $\mathcal{V}_i$, морфизмы -- пары $(f_1, f_2)$, где $f_i \in Mor(\mathcal{V}_i)$. Отсюда есть забывающие функторы -- проекции $\pi_1, \pi_2: \mathcal{V}_\times \to \mathcal{V}_i$\\
каждый из которых забывает соответственно второй или первый элементы в парах:

на объектах $\pi_i: (V_1, V_2) \to V_i$

на стрелках $\pi_i: (f_1, f_2) \to f_i$

%Условно $f_i$ хранит информацию о том, в какое из пространств он смотрит. Забывая одну из стрелок тройки, получаем пару, которая хранит само пространство $X$ и только одну из стрелок, т.е. только одно пространство $V_i$.

%Пусть $V_1, V_2$ -- ВП, тогда их прямое произведение $V_1 \times V_2, \pi_1, \pi_2$ -- это новое ВП вместе с двумя стрелками -- проекциями $\pi_i: V_1 \times V_2 \to V_i$ на $i$-ый сомножитель.

\underline{Утверждение}. Декартово произведение $\mathcal{V}_\times$ вместе с проекциями $\pi_1, \pi_2$ обладает универсальным свойством для пар функторов $F_i: X \to V_i$ из произвольной категории $\mathcal{X}$.

Диаграмма:

\[ \begin{diagram}
	\node[2]{\mathcal{X}}
		\arrow{s,r}{H}
		\arrow{sw,t}{F_1}
		\arrow{se,t}{F_2}
	\\
	\node{\mathcal{V}_1}
	\node{\mathcal{V}_1 \times \mathcal{V}_2}
		\arrow{w,b}{\pi_1}
		\arrow{e,b}{\pi_2}
	\node{\mathcal{V}_2}	
\end{diagram} \]

Возьмём произвольную пару функторов $F_1, F_2$. Тогда их можно пропустить через $\mathcal{V}_\times = \mathcal{V}_1 \times \mathcal{V}_2$ с соответствующими проекциями:

$$ \forall F_1, F_2 : \; \exists H = (F_l, F_r):$$
$$ F_l = \pi_1 \circ H, \; F_r = \pi_2 \circ H $$

Если $\mathcal{X}, \mathcal{V}_1, \mathcal{V}_2 = Vct$, то получаем универсальные объекты -- декартовы произведения векторных пространств.\\
Если к тому же обратить стрелки (вместо проекций будут вложения), то получим прямую сумму ВП.






\subsection*{Базисы векторных пространств}

\bigskip

К этому же универсальному элементу -- паре $(u, X)$ -- можно было подойти иначе. Для фикс. множества $X$ и поля $\Bbbk$ построить категорию
\begin{itemize}
\item объекты -- пары (V, f) -- произв. векторные пространства над данным полем с отображениями $f: X \to V$

\item стрелки -- линейные отображения ВП, сохраняющие отображения $f$.
\end{itemize}

Тогда пара $(X, u)$ обладает универсальным свойством в этой категории и поэтому является начальным объектом в ней.
